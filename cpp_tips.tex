\documentclass[a4paper,article,14pt]{extarticle}
\usepackage{spbudiploma_tempora}
\usepackage{euscript}
\usepackage{longtable}
\usepackage{makecell}
\usepackage[pdftex]{graphicx}
\usepackage{amsthm,amssymb, amsmath}
\usepackage{textcomp}
\usepackage{float}
\usepackage{listings}


\begin{document}

\tableofcontents
\pagebreak

\newpage
\specialsection{Введение}

\newpage
\section{Основы}
\subsection{Переменные}
\subsubsection{Основные типы}
В Unreal Engine C++ могут использоваться как и стандартные типы языка C++, так и инициализированные типы языка Blueprints. В данной таблице содержаться основные типы, которые используются при разработке на языке чертежей:

\begin{center}
\begin{tabular}{ |c|c|c| } 
 \hline
Название & Синтаксис C++ & Include \\ 
 \hline
Integer & int32 & - \\ 
 \hline
Float & float & - \\ 
\hline
Vector & FVector & \#include "Math/Vector.h" \\ 
\hline
Rotator & FRotator & \#include "Math/Rotator.h" \\ 
\hline
Text & FText & \#include "Internationalization/Text.h" \\ 
 \hline
Actor & AActor & \#include "GameFramework/Actor.h" \\ 
 \hline
Static Mesh & UStaticMesh & \#include "Engine/StaticMesh.h"\\ 
 \hline
Skeletal Mesh & USkeletalMesh & \#include "Engine/SkeletalMesh.h" \\
 \hline
Texture2D & UTexture2D & \#include "Engine/StaticMesh.h"\\ 
 \hline
Material & UMaterial & \#include "Engine/Texture2D.h"\\  
\hline
\end{tabular}
\end{center}

\subsubsection{Модификаторы}
Чтобы дать переменной в C++ особые свойства по типу - видимость в Unreal Engine Blueprints и так далее используется специальная конструкция - \textbf{UPROPERTY(<Params>)}. Вот основные параметры которые могут подаваться на вход:

\begin{itemize}
\item{BlueprintReadWrite - полная видимость в Blueprints}
\item{EditAnywhere - аналогия EditOnSpawn}
\item{Category="Name" \- - добавляет переменную в категорию Name}
\end{itemize}

Синтаксис:

\begin{lstlisting}[
    basicstyle=\small, %or \small or \footnotesize etc.
]              

{
      UPROPERTY(BlueprintReadWrite, Category="Params")
      int32 SIZE_X = 0;
};

\end{lstlisting}          

\subsubsection{Массивы}

Для создания массивов в Unreal Engine CPP используется - \href{https://docs.unrealengine.com/4.26/en-US/ProgrammingAndScripting/ProgrammingWithCPP/UnrealArchitecture/TArrays/}{\textbf{TArray<TYPE>}}, где TYPE - тип элементов массива.

Синтаксис:

\begin{lstlisting}[
    basicstyle=\small, %or \small or \footnotesize etc.
]              

{
      TArray<int32> indices;
      TArray<APuzzle*> puzzles;
};

\end{lstlisting}          

Основные операции с массивом:

\begin{itemize}
\item{Arr.Num() - длина массива}
\item{Arr[Index] - элемент массива}
\item{Arr.Add(Element) - добавить элемент массива}
\item{Arr.RemoveAt(Index) - удаляет элемент по индексу}
\end{itemize}

\subsection{Функции}



\newpage
\section{Structs}
\subsection{Создание структуры}

Для создания структуры используется пустой экземпляр класса CPP \textbf{None}. Из сгенерированных файлов необходимо удалить файл \textbf{<Name>.cpp} и оставить только \textbf{<Name>.h}. В Header файла создается следующая структура:
\\
\lstset{language=C}          
\begin{lstlisting}[
    basicstyle=\small, %or \small or \footnotesize etc.
]              
#pragma once

#include "<Name>.generated.h"

USTRUCT(BlueprintType)
struct F<Name> : public FTableRowBase
{
	GENERATED_BODY()
    
	FORCEINLINE F<Name>();
};

FORCEINLINE F<Name>::F<Name>(){
}
\end{lstlisting}          

\newpage
\section{AActor}
\subsection{Spawn Actor}
Для спауна эктора используется метод - \href{https://docs.unrealengine.com/4.26/en-US/ProgrammingAndScripting/ProgrammingWithCPP/UnrealArchitecture/Actors/Spawning/}{\textbf{SpawnActor()}}

Синтаксис:

\begin{lstlisting}[
    basicstyle=\small, %or \small or \footnotesize etc.
]   
{
      FVector location = FVector();
      FRotator rotation = FRotator();
      FActorSpawnParameters params;
      AActor *actor = (AActor*) GetWorld()->SpawnActor(
	AActor::StaticClass(),\&location,\&rotation,params);
		
};
\end{lstlisting}         


\end{document}